\subsection{Meta-heurísticas e busca tabu}
Uma meta-heurística, é uma forma heurística de resolução de problemas genéricos de otimização\cite{metaheuristicas2009}. Isto é, a meta-heurística é uma espécie de framework que pode ser utilizado para resolver diversos problemas considerados difíceis (problemas NP-árduos como o PCV supracitados são exemplos).

A meta-heurística se diferencia da heurística no que se diz respeito a utilização. A primeira é utilizada de modo geral, em diversos tipos de problemas de otimização, enquanto a segunda é específica para um problema. Como exemplos temos a meta-heurística Simulated Annealing e a heurística da inserção da aresta mínima (Problema do Caixeiro Viajante).

A meta-heurística escolhida para o presente trabalho foi a busca tabu.Essa funciona da maneira que segue:

Uma solução heurística para o problema é encontrada e, sobre esta, aplicamos as operações de vizinhança, como em uma busca local. Entretanto, a busca tabu utiliza de um artifício para não causar um grande número de repetições nas soluções vizinhas encontradas (o que acontece com a busca local).

Esse artifício é a utilização de uma tabela, ou uma lista tabu, na qual são armazenados os movimentos já realizados recentemente, os quais serão "proibidos" de se repetirem. Assim, causamos uma diversificação e uma intensificação maior nos resultados.

\subsection{A busca tabu aplicada ao PCV}


O algoritmo implementado aplica a busca tabu ao Problema do Caixeiro Viajante. O PCV, como já citado anteriormente, é um problema NP-Difícil, portanto é interessante, para aproximarmos os resultados, o uso de meta-heurísticas.

A primeira parte a ser entendida é a solução inicial. Nesse projeto, a solução inicial foi obtida através da utilização da heurística do vizinho mais próximo para cada um dos vértices como inicial. Ou seja, utilizamos essa heurística $n$ vezes, e encontramos, dentre os resultados, o de menor custo.

Após isso, temos a operação de vizinhança escolhida. Dada uma solução (representada por um vetor de inteiros - a sequencia de vértices do ciclo), ela consiste em realizar permutações de modo que, inicialmente, o primeiro nó troque com outro se a solução gerada tiver custo menor do que a solução original (da qual queremos gerar o conjunto de vizinhos). Sendo assim, ao iniciar a operação temos um nodo de troca ($V_k$) setado para o primeiro nó do ciclo inicial. A partir do momento em que ocorre essa troca, setamos o novo nodo de troca ($V_k$) para o que trocamos com o inicial. Esse ciclo de trocas é repetido $n$ vezes.

O algoritmo da busca tabu em si, gera a vizinhança da solução inicial e, para cada vizinho gerado, gera-se a vizinhança deste, selecionando das $n$ vizinhanças, uma solução de menor custo.

A lista tabu vai incluir os movimentos de troca, com seus respectivos vértices. Por exemplo, ao trocarmos o vértice 3 com o 8, teremos, na lista tabu, o vetor [3,8], simbolizando esse movimento, que não poderá ser repetido até a lista atingir o seu tamanho máximo, quando o primeiro elemento da lista é retirado para a adição de outro no fim desta.